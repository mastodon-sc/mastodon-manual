\section{Getting started with Mastodon.}

This tutorial is the starting point for new Mastodon users. 
It will walk you through basic operations in Mastodon, opening a dataset and creating a Mastodon project, automatically detect cells and link them, and show you how to use the main views of Mastodon.
We don't go into details, and will revisit the features we survey here later.

Mastodon was created specially because we needed to harness very big, multi-view images. We wanted to generate  comprehensive lineages and follow a large number of cells over a very long time.
This accumulation of inflated words is tied to the very large  - in objective disk space  occupation - images we deal with using modern microscopy tools. 
Such datasets might not be optimal for a first tutorial with Mastodon.
So just for this tutorial we will use a smaller dataset.
It is a small region cut into a movie following the development of a drosophila, acquired in Pavel Tomancak lab (MPI-CBG).
You can find it on Zenodo\footnote{\href{https://zenodo.org/record/3336346}{https://zenodo.org/record/3336346}} there: \href{https://doi.org/10.5281/zenodo.3336346}{\includegraphics[height=1.4\fontcharht\font`\B]{figures/zenodo3336346.png}}

It is a zip file that contains 3 files:
\begin{minted}[bgcolor=lightgray]{text}
    14M  datasethdf5.h5
   2.7K  datasethdf5.settings.xml
   8.7K  datasethdf5.xml
\end{minted}
